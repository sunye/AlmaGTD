\section{Interface Homme Machine}

\begin{frame}
\frametitle{IHM : Conception et int\'egration}
\begin{itemize}
  \item Deux probl\'ematiques
  \item La r\'ealisation des acc\`es aux fonctionnalit\'es de la m\'ethode GTD
  \item Une conception centr\'ee utilisateur et ergonomique
\end{itemize}
\end{frame}

\begin{frame}
\frametitle{Les choix d'impl\'ementation, architecture logicielle}
\begin{itemize}
\item Notre application \'etant d\'evelopp\'ee en java, nous avons d\'ecid\'e d'utiliser la librairie SWING utilis\'ee en TP d'IHM.
\item Nous avons d\'evelopp\'e l'IHM selon le patron de conception MVC.
\item S\'eparation du Mod\`ele, de la Vue et du Contr\^oleur (sp\'ecification
des composants, conception des interfaces)
\item Pour pouvoir d\'evelopper l'IHM en parall\`ele de l'application nous
avons r\'ealis\'e des bouchons pour simuler le comportement du contr\^oleur.
\end{itemize}
\end{frame}

\begin{frame}
\frametitle{Les points mis en valeur dans notre IHM, crit\`eres ergonomiques}
\alert{Guidage de l'utilisateur} : Groupement entre informations relatives au panier, au contexte ou au projet.\\
\alert{Densit\'e informationnelle} r\'eduite gr\^ace \^a l'utilisation d'onglet.\\
\alert{Contr\^ole explicite} : message d'information en cas de mauvaise utilisation des fonctionnalit\'es\\
\alert{Codes simples} : "+" pour ajout.\\
\alert{Homog\'en\'eit\'e, coh\'erence}\\
\alert{Correction des erreurs}\\
\end{frame}


\begin{frame}
\frametitle{Le compromis dans la r\'ealisation}
\begin{itemize}
  \item Vision des concepts GTD dans l'interface
  \item Acc\'es rapides aux fonctionnalit\'es
\end{itemize}
\end{frame}
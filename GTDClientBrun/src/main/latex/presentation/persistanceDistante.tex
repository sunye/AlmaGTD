\section{Persistance distante}

\begin{frame}
\frametitle{Persistance distante, source de complexit�}
%Nous avions � faire : Communication corba avec le serveur.\\
\textbf{Probl�mes rencontr�s :}
\begin{itemize}
\item Mauvaise �valuation du temps.
\item Mauvaise synchronisation avec les autres groupes.%On a recu les idl tard
\item Probl�me de coh�sion des mod�les.
\end{itemize}
\textbf{Cons�quence sur GTD :}
\begin{itemize}
\item Fonctionnalit�s non impl�ment�es dans GTD.
\end{itemize}


\end{frame}

\begin{frame}[fragile]
\frametitle{Solution de persistance sur un serveur distant via CORBA}
\begin{itemize}
\item Impl�mentation de l'interface IPersistance utilis�e dans la persistance locale. Voici les fonctions que nous aurions impl�ment�es pour chaque �l�ment persistant :
\lstset{breaklines=true, basicstyle=\footnotesize ,  language=java}
\begin{lstlisting}
public List<Idee> findAllIdees();
public Idee findIdee(final Integer id);
public Boolean update(final Idee entity);
public Boolean deleteAllIdees();
public Boolean delete(final Idee entity);
\end{lstlisting}
\end{itemize}
\end{frame}


\begin{frame}[fragile]
\frametitle{Solution de persistance sur un serveur distant via CORBA}
\begin{itemize}
\item Patron de conception proxy pour d�finir l'ensemble de classe repr�sentant les classes du mod�le. Les instances de ces classes seraient des repr�sentations des objets corba distants.
\item Ces classes contiendraient un IOR permettant de trouver l'objet distant.
\item Regroupement des op�rations serait dans une classe unique, seul le type d'objet que l'on fait persister varie.
\item Param�trage de cette classe unique par un type T repr�sentant le type d'objet � persister.\\
Signature des classes :
\lstset{breaklines=true, basicstyle=\footnotesize ,  language=java}
\begin{lstlisting}
abstract class ObjetDistant<T extends GTDEntity>
class TacheImpl extends ObjetDistant<Tache> implements Tache
\end{lstlisting}
\item Pour communiquer avec le serveur nous aurions besoin de l'adresse du registre CORBA qui serai dans un attribut.
\end{itemize}
%Nous pouvons ainsi assurer la repr�sentation en locale d'objet distants qu'on peut sauvegarder puis r�cup�rer. Il faut toutefois avoir recours � un m�canisme d'identification aupr�s du serveur. La session d'identification doit �tre initi�e aupr�s du serveur d�s le d�marrage de l'application, cette connexion doit �tre maintenue afin de permettre toute instanciation tardive.

\end{frame}


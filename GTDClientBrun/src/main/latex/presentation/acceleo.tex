\section{G\'en\'eration de code}

\begin{frame}
\frametitle{De l'UML au code source...}
\begin{itemize}
\item volont\'e de fructifier le travail de mod\'elisation
\item int\'egration du mod\`ele dans le processus de d\'eveloppement (MDA)
\end{itemize}
\end{frame}

\begin{frame}
\frametitle{\'Ecriture de scripts Acceleo}
Impl\'ement\'e :
\begin{itemize}
\item interfaces
\item classes, h\'eritages et r\'ealisations
\item attributs internes et relations (cardinalit\'es)
\item m\'ethodes internes, setters, getters, contractuelles (visibilit\'es)
\end{itemize}
Potentiellement :
\begin{itemize}
\item annotations et fichiers de configuration JPA
\item classes st\'er\'eoptyp\'ees DAO, Singleton, Etat
\end{itemize}
\end{frame}

\begin{frame}
\frametitle{Bilan}
\begin{itemize}
\item[-] langage difficile \`a adopter (API de r\'eflexion ?)
\item[-] p\'er\'enit\'e du mod\`ele ?
\item pertinent pour les classes donn\'ees, et le reste ?
\item[+] gain de temps sur le long terme (p\'er\'enisation des scripts)
\item[+] coh\'erence mod\`ele-code pr\'eserv\'ee
\item[+] plus grande agilit\'e du mod\`ele
\end{itemize}
\end{frame}

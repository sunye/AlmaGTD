\chapter{Introduction}


Dans le cadre du projet multi-modules en master 2 ALMA, il nous a été demandé de modéliser et concevoir un logiciel permettant de planifier nos tâches au quotidien.
La méthode Getting Things Done développée par David Allen se présente sous la forme d'une méthode de gestion des priorités quotidiennes. Cette première partie présente l'analyse de cette méthode. Nous nous interressons dans cette partie à l'analyse de la discipline GTD : ce premier livrable ne concerne donc pas l'application fournissant les fonctionnalités de GTD, mais bien la méthode GTD elle même. Le langage UML (Unified Modeling Language) permet de présenter cette analyse à l'aide des documents suivants :\\

\begin{itemize}


\item Un dictionnaire de données qui permettra d'identifier et de détailler l'ensemble des concepts identifiés,
\item un cas d'utilisation général ainsi qu'un sous ensemble de cas modélisés à l'aide d'une description textuelle selon le canevas de CockBurn,
\item des diagrammes de scénarios afin d'illustrer les interactions entre l'utilisateur et le système,
\item des instantanés illustrant l'impact des cas d'utilisation sur les objets,
\item un diagramme d'activités représentant le fonctionnement général de la méthode,
\item un diagramme de classes (niveau analyse),
\item un ensemble de contraintes exprimées en OCL.

\end{itemize}



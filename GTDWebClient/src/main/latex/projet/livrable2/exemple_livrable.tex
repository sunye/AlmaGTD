\documentclass[a4paper, french, 10pt]{report}
\usepackage{alma}

\usepackage[utf8]{inputenc}
\usepackage[pdftex]{graphicx}
\usepackage[english, frenchb]{babel}

\usepackage{amsmath}
\usepackage{amssymb}
\usepackage{tipa}
\usepackage{textcomp}

\usepackage{fancyhdr}
\pagestyle{fancy}
\renewcommand{\chaptermark}[1]{\markboth{#1}{}}
\renewcommand{\sectionmark}[1]{\markright{\thesection\ #1}}
\fancyhf{}
\fancyhead{} % clear all footer fields 
\fancyhead[LE,RO]{\bfseries\thepage}
\fancyhead[LO]{\bfseries\rightmark}
\fancyhead[RE]{\bfseries\leftmark}

\fancyfoot{}
\fancyfoot[LE,RO]{\scriptsize{Jean, Marie \\ Charlie et les autres}} 
\fancyfoot[LO,CE]{\scriptsize\textsl{Université de Nantes\\Master Alma}} 
\fancyfoot[CO,RE]{\scriptsize{\today}}

\renewcommand{\headrulewidth}{0.4pt}
\renewcommand{\footrulewidth}{0.4pt}
\addtolength{\headheight}{0.5pt}
\setlength{\footskip}{0in}
\renewcommand{\footruleskip}{0pt}
\fancypagestyle{plain}{%
 \fancyhead{}
 \renewcommand{\headrulewidth}{0pt}
}

%
%\parindent 0in
\parskip 0.05in


\title{Titre du projet}
\author{Jean, Marie, Charlie et les autres}
\date{}




\begin{document}
\selectlanguage{frenchb}
\maketitle

\begin{abstract}
	résumé du rapport
\end{abstract}
 
\tableofcontents
 

\chapter{Analyse du problème}
 
\section{Exemple de code OCL}
\begin{ocl}
context Etudiant::age() : Integer
post correct: result = (today - naissance).years()

context Typename::operationName(param1: type1, ...): Type
post: result = ...

context Typename::operationName(param1: type1, ...): Type
post resultOk: result = ...
\end{ocl}

\section{Exemple d'un cas d'utilisation}

\subsection{Réaliser une affectation}

\begin{usecase}{Réaliser une affectation}

\begin{information}

\item[Goal in the context:]
Le chef de département réalise la confirmation d'un souhait d'un enseignant (validation) ou impose une intervention dans le département à un enseignant (affectation ``forçée''). L'enseignant devra se soumettre aux décisions du chef de département que ce soit pour une validation d'un souhait ou une affectation imposée.

\item[Scope:]
Département

\item[Level:]
Résumé

\item[Precondition:]
Le chef de département connait pour l'enseignant concerné : les souhaits (seulement ceux déjà publiés par l'enseignant), les possibles conflits et les affectations le concernant.

\item[Success End Condition:]
L'enseignant est affecté à un ou plusieurs enseignements (souhait validé ou intervention imposée). Le volume horaire total effectué par l'enseignant a été recalculé.

\item[Failed End Condition:]
L'affectation n'a pas été réalisée (l'enseignement est déjà affecté à un autre enseignant : conflit).

\item[Primary actor:]
Chef de département.

\item[Trigger:]
Demande de réalisation d'une affectation faite par le chef de département.\\

\end{information}

%\\
\begin{scenario}
\item Le chef de département demande au système de visualiser l'ensemble des souhaits des enseignants.
\item Le système renvoie les souhaits qui ont été publiés par les enseignants.
\item Le chef de département sélectionne des souhaits, selon certains critères (l'enseignant, le module concerné, \dots).
\item Le système renvoie les souhaits correspondants aux critères.  
\item Le chef de département choisit un v\oe u d'un enseignant afin de le valider.
\item Le système valide le v\oe u, l'enseignement associé au v\oe u est affecté à l'enseignant concerné : c'est-à-dire que le v\oe u donne lieu à une intervention effective.\\
\end{scenario}

%\\
\begin{extension}
\item[5a] Le chef de département choisit d'imposer une intervention à un enseignant dans son département (affectation imposée).
\item[5a1] Le système affecte l'intervention à l'enseignant.
\item[5b] L'utilisateur choisit de valider une demande d'intervention extérieure ou une demande spéciale (remplacement, encadrement de stage, congés, \dots) de l'enseignant.
\item[5b1] Le système affecte l'intervention ou valide la demande spéciale de l'enseignant sans créer de conflits avec les autres enseignants.\\
\end{extension}
\end{usecase}

\section{Exemple d'un dictionnaire de données}
	
	\begin{tabular}{|p{3cm}|p{6cm}|p{2cm}|p{4cm}|}

\hline \textbf{Notion} & \textbf{Définition} &\textbf{Traduit en} & \textbf{Nom informatique} \\

\hline Affectation & Action de déterminer une \textbf{intervention}, i.e. d'associer un \textbf{enseignant} à un \textbf{enseignement} donné. C'est le \textbf{chef de département} qui est chargé de déterminer les différentes affectations en fonction, le plus souvent, des \textbf{v\oe ux} réalisés par les divers \textbf{enseignants}. & &\\

\hline Chef de département & Acteur du système. Il est le responsable d'un \textbf{département} : il gère les \textbf{modules} (ainsi que les \textbf{enseignements} associés), les \textbf{enseignants} et leurs \textbf{interventions} pour son \textbf{département}. & Acteur & ChefDepartement \\

\hline Conflit & Fait que deux \textbf{v\oe ux} soit imcompatibles (i.e. que deux \textbf{enseignants} aient émis les mêmes choix concernant un ou des \textbf{enseignements}). C'est au \textbf{chef de département} de régler les conflits en réalisant les affectations. & &\\

\hline Contrat de service & Nombre d'heures minimum (et parfois nombre d'heures maximum) d'\textbf{enseignements} à effectuer pour un \textbf{enseignant} donné. Il est indépendant des \textbf{départements} dans lequels l'\textbf{enseignant} intervient : il est unique et seulement déterminé par le statut de l'\textbf{enseignant}. & Classe & ContratDeService \\

\hline Département & Entité administrative (d'une université) identifiée par un nom. Il comprend un ensemble de \textbf{modules} et d'\textbf{enseignants} qui lui sont rattachés. Chaque département a pour responsable un \textbf{chef de département}. Plusieurs \textbf{enseignants} peuvent donner des \textbf{enseignements} pour le compte de chaque \textbf{département}. & Classe & Departement  \\

\hline Enseignant & Personne "physique" travaillant pour le compte d'un \textbf{département} et identifiée par son nom, son prénom et son statut. Un enseignant peut "intervenir" dans différents \textbf{départements} pour dispenser un certain nombre d'\textbf{enseignements}. C'est un également un acteur du sytème puiqu'il peut effectuer des \textbf{v\oe ux} concernant les \textbf{enseignements} qu'il désire donner. & Classe et Acteur & Enseignant \\

\hline
\end{tabular}

\chapter{Spécification d'exigences logicielles}
% Template proposé par "The Unified Process for EDUcation (UPEDU)"
% http://www.yoopeedoo.org/upedu/

%Selon le wikipedia \url{http://fr.wikipedia.org/wiki/Exigence_(ingénierie)}
%De bonnes exigences doivent être :
% * Nécessaires – Elles doivent porter sur des éléments nécessaires, c'est-à-dire des éléments importants du système que d'autres composants du système ne pourraient pas compenser.
% * Non ambiguës – Elles doivent être susceptibles de n'avoir qu'une seule interprétation.
% * Concises – Elles doivent être énoncées dans un langage qui soit précis, bref et agréable à lire, et qui de plus communique l'essence de ce qui est exigé.
% * Cohérentes – Elles ne doivent pas contredire d'autres exigences établies, ni être contredites par d'autres exigences. De plus, elle doit, d'un énoncé d'exigence au suivant, utiliser des termes et un langage qui signifie la même chose.
% * Complètes – Elles doivent être énoncées entièrement en un endroit et d'une façon qui ne force pas le lecteur à regarder un texte supplémentaire pour savoir ce que l'exigence signifie.
% * Accessibles – Elles doivent être réalistes quant à aux moyens mis en œuvre en termes d'argent disponible, avec les ressources disponibles, dans le temps disponible.
% * Vérifiables – Elles doivent permettre de déterminer si elles ont été atteintes ou non selon l'une de quatre méthodes possibles : inspection, analyse, démonstration, ou test.

\section{Introduction}
L’introduction donne une vue d’ensemble de tout le document. On y présente toute information que le lecteur a besoin pour  comprendre le document. Elle comprend l’objectif du document, sa portée, les définitions, acronymes et abréviations, les références et une vue d’ensemble du document.
Note : La SEL comporte l’ensemble des exigences logicielles pour une portion ou pour tout le système. La présente spécification est adaptée pour  un projet utilisant une modélisation de cas d’utilisation. Cet artéfact est un paquetage qui comprend les cas d’utilisation du modèle des cas d’utilisation et les spécifications supplémentaires applicables ainsi que les autres informations pertinentes.
Plusieurs aménagements d’une SEL sont possibles. La norme (IEEE830-1998) est la référence pour de plus amples explications ainsi que pour d’autres options d’organisation du document.

\subsection{Objectif}
Préciser les objectifs de ce chapitre. La SEL doit décrire le comportement externe de l’application ou du sous-système identifié. Elle décrit aussi les exigences non-fonctionnelles et les autres facteurs nécessaires à une description complète et compréhensible des exigences pour le logiciel.

	\subsection{Conventions}
%<Describe any standards or typographical conventions that were followed when writing this SRS, such as fonts or highlighting that have special significance. For example, state whether priorities  for higher-level requirements are assumed to be inherited by detailed requirements, or whether every requirement statement is to have its own priority.>

	\subsection{Audience}
%<Describe the different types of reader that the document is intended for, such as developers, project managers, marketing staff, users, testers, and documentation writers. Describe what the rest of this SRS contains and how it is organized. Suggest a sequence for reading the document, beginning with the overview sections and proceeding through the sections that are most pertinent to each reader type.>

	\subsection{Portée du document}
Une brève description de la portée de ce document, l’application qu’il décrit, les caractéristiques ou autres sous-systèmes auxquels l’application est associée, le ou les modèles de cas d’utilisation qu’il décrit ainsi que tout autre chose qui peut être influencée ou affectée par ce document.

	\subsection{Définitions, acronymes et abréviations}
	Énumérer les définitions de tous les termes, acronymes et abréviations nécessaires à la compréhension du document d’architecture logicielle. Cette information peut renvoyer à l’artéfact Glossaire du projet..


	\subsection{Références}
Cette section comporte la liste de tous les documents cités dans le document. Chaque document doit être identifié par son titre, son numéro, lorsque applicable, sa date et l’organisation qui l’a publiée. Les sources qui peuvent fournir les références doivent être citées. Cette dernière information peut être elle-même une référence à une annexe ou à un autre document.

	\subsection{Organisation du chapitre}
Cette section décrit le contenu du reste du document  et explique comment le document est organisé.

\section{Description générale}
Décrire les principaux facteurs qui affectent le produit et ses exigences. On n’y énonce pas des exigences spécifiques, mais on y fournit une toile de fond aux exigences qui sont définies en détail à la section 3 afin d’en faciliter la compréhension. Cela comprend les items suivants:

	\subsection{Perspectives du produit}
%<Describe the context and origin of the product being specified in this SRS. For example, state whether this product is a follow-on member of a product family, a replacement for certain existing systems, or a new, self-contained product. If the SRS defines a component of a larger system, relate the requirements of the larger system to the functionality of this software and identify interfaces between the two. A simple diagram that shows the major components of the overall system, subsystem interconnections, and external interfaces can be helpful.>

	\subsection{Fonctions du produit}
Décrire brièvement les fonctions principales du logiciel. Par exemple : 
Les fonctions d’un système de gestion peuvent être la maintenance d’un compte client, accéder à l’état de compte du client et produire la facturation.

	
	\subsection{Caractéristiques et classes d'utilisateurs}
Décrire les caractéristiques générales des utilisateurs qui ont un impact sur les exigences du document. Cela inclut le niveau de scolarité, l’expérience et l’expertise technique.
%<Identify the various user classes that you anticipate will use this product. User classes may be differentiated based on frequency of use, subset of product functions used, technical expertise, security or privilege levels, educational level, or experience. Describe the pertinent characteristics of each user class. Certain requirements may pertain only to certain user classes. Distinguish the favored user classes from those who are less important to satisfy.>

	\subsection{Environnement opérationnel}
%<Describe the environment in which the software will operate, including the hardware platform, operating system and versions, and any other software components or applications with which it must peacefully coexist.>

	\subsection{Contraintes de conception et d'implémentation}
%<Describe any items or issues that will limit the options available to the developers. These might include: corporate or regulatory policies; hardware limitations (timing requirements, memory requirements); interfaces to other applications; specific technologies, tools, and databases to be used; parallel operations; language requirements; communications protocols; security considerations; design conventions or programming standards (for example, if the customer’s organization will be responsible for maintaining the delivered software).>

	\subsection{Documentation utilisateur}
%<List the user documentation components (such as user manuals, on-line help, and tutorials) that will be delivered along with the software. Identify any known user documentation delivery formats or standards.>

	\subsection{Hypothèses et dépendances}
Décrire tout élément de faisabilité technique, disponibilité de sous-système ou de composant ou toute autre hypothèse liée au projet de laquelle dépend la viabilité du logiciel.
	
%<List any assumed factors (as opposed to known facts) that could affect the requirements stated in the SRS. These could include third-party or commercial components that you plan to use, issues around the development or operating environment, or constraints. The project could be affected if these assumptions are incorrect, are not shared, or change. Also identify any dependencies the project has on external factors, such as software components that you intend to reuse from another project, unless they are already documented elsewhere (for example, in the vision and scope document or the project plan).>

\subsection{Exigences reportées}
Énumérer les exigences qui peuvent être réalisées dans des versions futures du système.



\section{Fonctionnalités du logiciel}
Décrire les exigences fonctionnelles du système qui peuvent être exprimées et langage naturel. Pour plusieurs applications, c’est la partie principale de la SEL et son organisation doit, par conséquent, être bien réfléchie. Elle est habituellement hiérarchisée par caractéristiques, mais elle peut l’être, par utilisateur ou par sous-système. Les exigences fonctionnelles peuvent inclure les caractéristiques, les capacités et la sécurité.

Lorsque des outils de développement, tels des référentiels d’exigences ou des outils de modélisation sont utilisés, on peut référer à ces données en indiquant l’endroit et le nom de cet outil]

	\subsection{Cas d'utilisation num. 1}
% N'utilisez pas le nom "Cas d'utilisation 1", mais le nom du cas d'utilisation en quelques mots.

	\subsubsection{Description}
% Utilisez le canevas de cockburn pour décrire le cas d'utilisation
% Indiquez la priotité: Haute, Moyenne et Basse
% Indiquez, éventuellement, des mesures spécifiques, comme le profit, le coût, les risques, etc. 
% (sur une échelle relative d'un minimum de 1 à un maximum de 9))

	\subsubsection{Exigences fonctionnelles}
%<Itemize the detailed functional requirements associated with this feature. These are the software capabilities that must be present in order for the user to carry out the services provided by the feature, or to execute the use case. Include how the product should respond to anticipated error conditions or invalid inputs. Requirements should be concise, complete, unambiguous, verifiable, and necessary. Use “TBD” as a placeholder to indicate when necessary information is not yet available.>

%<Each requirement should be uniquely identified with a sequence number or a meaningful tag of some kind.>

%REQ-1:	
%REQ-2:	



\section{Exigences des interfaces externes}
	\subsection{Interface utilisateur}
%<Describe the logical characteristics of each interface between the software product and the users. This may include sample screen images, any GUI standards or product family style guides that are to be followed, screen layout constraints, standard buttons and functions (e.g., help) that will appear on every screen, keyboard shortcuts, error message display standards, and so on. Define the software components for which a user interface is needed. Details of the user interface design should be documented in a separate user interface specification.>	

	\subsection{Interface matérielle}
%<Describe the logical and physical characteristics of each interface between the software product and the hardware components of the system. This may include the supported device types, the nature of the data and control interactions between the software and the hardware, and communication protocols to be used.>

	\subsection{Interface logicielle}
%<Describe the connections between this product and other specific software components (name and version), including databases, operating systems, tools, libraries, and integrated commercial components. Identify the data items or messages coming into the system and going out and describe the purpose of each. Describe the services needed and the nature of communications. Refer to documents that describe detailed application programming interface protocols. Identify data that will be shared across software components. If the data sharing mechanism must be implemented in a specific way (for example, use of a global data area in a multitasking operating system), specify this as an implementation constraint.>

	\subsection{Interfaces ou protocoles de communication}
%<Describe the requirements associated with any communications functions required by this product, including e-mail, web browser, network server communications protocols, electronic forms, and so on. Define any pertinent message formatting. Identify any communication standards that will be used, such as FTP or HTTP. Specify any communication security or encryption issues, data transfer rates, and synchronization mechanisms.>

	\subsection{Contraintes de mémoire}

	\subsection{Hypothèses et dépendances}

\section{Autres exigences non-fonctionnelles}

Décrire les exigences qui ne sont pas incluses dans les cas d’utilisation ainsi que les exigences non-fonctionnelles. On peut référer au document de Spécifications supplémentaires.

	\subsection{Utilisabilité} 
	Décrire les exigences qui affectent l’utilisabilité comme, par exemple:
	\begin{itemize}
		\item Le temps de formation nécessaire à un utilisateur normal ou expert avant d’être productif.
		\item Les temps d’exécution pour les tâches courantes
		\item Les exigences pour satisfaire aux standards d’utilisabilité d’interface graphique de, par exemple, Microsoft.
	\end{itemize}
	
	\subsubsection{Nom de l’exigence d’utilisabilité 1}
	Description de l’exigence 

	\subsection{Fiabilité }
	
	Décrire les exigences qui affectent la fiabilité comme, par exemple:
	\begin{itemize}

	\item	La disponibilité: le pourcentage d’heures d’utilisation, les périodes de maintenance, mode d’opération lors de dégradation, etc.
	\item	Durée moyenne de fonctionnement avant défaillance, exprimée en heures, en jours, en mois ou en années.
	\item	Durée moyenne de rétablissement, qui est le délai moyen de réparation d'une unité fonctionnelle après une défaillance.
	\item	Exactitude. précision, souvent définie par de normes, requise pour les extrants.
	\item	Nombre maximum d’anomalies exprimé habituellement en KLOC, en défaut par millier de ligne de code ou par points de fonction
	\item	Criticité d’anomalie, mineure, significative, critique en décrivant ce que critique signifie. 
	 
	\end{itemize}
	
		\subsubsection{Nom de l’exigence de fiabilité 1}
		Description de l’exigence
		
	\subsection{Exigences de performance}
	Décrire les caractéristiques de la performance du système. Référer les cas d’utilisation lorsque applicable.

	\begin{itemize}
	\item	Temps de réponse par transaction (moyen, maximum)
	\item	Débit (transactions par seconde)
	\item	Capacité (nombre de client ou de transaction que le système doit supporter)
	\item	Mode d’opération lors de dégradation (Mode d’opération acceptable lorsque la performance du système se détériore)
	\item	Utilisation de ressources (mémoire, disque, communications, etc.
	\end{itemize}
	
	\subsubsection{Nom de l’exigence de performance 1}
	Description de l’exigence

	\subsection{Maintenabilité}
	
	Décrire les exigences qui permettent d’assurer le support et la maintenabilité du système comme, par exemple, les normes de codage, le conventions d’identification, les bibliothèques de classe, l’accès à la maintenance, les services de maintenances, etc. 

	\subsubsection{Nom de l’exigence de maintenabilité 1}
	Description de l’exigence

	\subsection{Exigences de sûreté}
%<Specify those requirements that are concerned with possible loss, damage, or harm that could result from the use of the product. Define any safeguards or actions that must be taken, as well as actions that must be prevented. Refer to any external policies or regulations that state safety issues that affect the product’s design or use. Define any safety certifications that must be satisfied.>

	\subsection{Exigences de sécurité}
	Identifier les données qui doivent être protégées et le type de menace auquel elles sont exposées comme, par exemple, menaces physiques, types de personnes qui peuvent être la source de menace, les exigences d’accès au système, l’encryptage des données, la vérifiabilité qui est la détection des anomalies et des opérations frauduleuses. 

	Énumérer la liste des objets qui doivent faire l’objet d’une protection physique ou logique

	\subsection{Attributs de qualité logicielle}
	
%<Specify any additional quality characteristics for the product that will be important to either the customers or the developers. Some to consider are: adaptability, availability, correctness, flexibility, interoperability, maintainability, portability, reliability, reusability, robustness, testability, and usability. Write these to be specific, quantitative, and verifiable when possible. At the least, clarify the relative preferences for various attributes, such as ease of use over ease of learning.>

\section{Autres exigences}
%<Define any other requirements not covered elsewhere in the SRS. This might include database requirements, internationalization requirements, legal requirements, reuse objectives for the project, and so on. Add any new sections that are pertinent to the project.>


\section{Classification des exigences fonctionnelles}

Énumérer dans un tableau toutes les exigences fonctionnelles et leur type, essentielle, souhaitable ou optionnelle. Elles peuvent être triées par leur ordre d’apparition dans le document ou par type.

\begin{center}
\begin{tabular}{|c|c|c|}
\hline
Code & Exigence & Type\\
\hline
REQ-1 & exigence 1 & optionnelle\\
\hline
REQ-2 & exigence 2 & essentielle\\
\hline
\end{tabular}
\end{center}

\chapter{Architecture}
 
% your text here
 
\chapter{Spécification des composants}
 
% your text here
 
\chapter{Conception détaillée}
 
% your text here
 
\chapter{Implémentation}
 
% your text here


\begin{thebibliography}{99}
  
\bibitem{baba_books}
Books of Shrii Shrii Anandamurti (Prabhat Ranjan Sarkar): \\
http://shop.anandamarga.org/books/sarkar/eledit70.htm
 
\bibitem{anandamitra}
Avtk. Ananda Mitra Ac., \emph{The Spiritual Philosophy of Shrii Shrii Anandamurti: A Commentary on Ananda Sutram}, Ananda Marga Publications (1991) \\
ISBN: 81-7252-119-7
 
\end{thebibliography}
  
\appendix
\chapter{Glossaire}
%<Define all the terms necessary to properly interpret the SRS, including acronyms and abbreviations. You may wish to build a separate glossary that spans multiple projects or the entire organization, and just include terms specific to a single project in each SRS.>
\end{document}
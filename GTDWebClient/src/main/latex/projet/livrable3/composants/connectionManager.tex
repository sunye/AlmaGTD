
\section{Composant ConnectionManager}


Ce composant s'occupe de gérer l'ensemble des sessions utilisateurs. 
En effet notre client Web pouvant être utilisé par une multitude de personnes, le composant s'occupe d'assurer 
l'identification vers le serveur GTD et/ou ToodleDo, la connexion entre notre
serveur Web et le serveur GTD et/ou ToodleDo. Lors d'une perte de connexion
entre le client et le serveur Web, il s'occupera d'envoyer une demande de
déconnexion vers le(s) serveur(s).

	\subsection{Gestion des comptes ToodleDo et du serveur GTD}
	L'application fonctionne avec différents serveurs. Ainsi, le choix qui à été fait consiste à regrouper les informations de chaque serveur (login, password) au sein d'un meta-compte. Celui-ci, dispose de son propre login et mot de passe. Il permet à l'utilisateur de saisir un seul couple login mot de passe pour se connecter aux serveurs ToodleDo et GTD. Les informations concernant ses comptes sont stockées localement sur le serveur d'application.

	\subsection{Liaison vers le client}
	La communication avec le client sera effectuée à l'aide de l'interface
	IGTDWebServer. Cette interface expose toutes les méthodes de
	ConnectionManager et ProcessManager. Le WebServer peut grâce à elle,
	rediriger les bons appels de connexion/déconnexion vers le composant
	ConnectionManager, et les appels métiers vers ProcessManager.


	\subsection{Liaison vers DataManager}
	La communication avec le composant DataManager est effectuée par
	l'implémentation des méthodes décrites par l'interface du composant ConnectToServer.
	
	\subsection{Test des connexions et gestion des erreurs}
	Ce composant doit permettre d'indiquer aux autres composants l'état des connexions en temps réel. Pour cela lors de la phase de 
	connexion, il teste la disponibilité des deux serveurs à savoir du serveur GTD
	et de ToodleDo. L'interception des exceptions permettra de modifier l'état des
	connexions à chaque instant. En effet, lorsqu'une exception sera levée de la part du DataManager, il en informera ses observateurs (dont ConnectionManager). Ce dernier mettra alors à jour les données de connexion.
